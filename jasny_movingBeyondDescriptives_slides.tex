\documentclass{beamer}
%\usefonttheme{professionalfonts}

\usepackage[round]{natbib}
\bibliographystyle{chicago}
\def\bibfont{\tiny}
\setlength{\bibsep}{0pt plus 0.3ex}

\usepackage{tikz}
\usetikzlibrary{positioning,shapes.geometric,graphs,scopes,backgrounds}
\usetikzlibrary{arrows}
\usetikzlibrary{shapes.arrows}
\usetikzlibrary{decorations.markings}
\usepackage{varwidth}
\usepackage{graphicx}
\usepackage{epstopdf}
\usepackage{multirow}



\usefonttheme{serif}
%\usefonttheme{structuresmallcapsserif}

%%%%%%%%%%%%%%%%

\title[Moving Beyond Descriptives]{Moving Beyond Descriptives}
\author[LJasny]{Lorien Jasny\inst{1}}
\institute[SESYNC]{\inst{1} University of Exeter\\
\texttt{L.Jasny@exeter.ac.uk}\\
				{\insertlogo}}
\logo{%
    \includegraphics[width=2cm,height=1.5cm,keepaspectratio]{../introToSNAinR/images/logo_exeter.jpg}~%
    \includegraphics[width=2cm,height=1.5cm,keepaspectratio]{../introToSNAinR/images/Q-Step-Logo.jpg}%
}

\date{\textit{EUSN Workshop}\\11 September 2022}
\usetheme{Hannover}
\def\swidth{1.6cm}
\setbeamersize{sidebar width left=\swidth}
\setbeamertemplate{sidebar left}
{
  {\usebeamerfont{title in sidebar}%
    \vskip1.5em%
    \usebeamercolor[fg]{title in sidebar}%
    \insertshorttitle[width=\swidth,center,respectlinebreaks]\par%
    \vskip1.25em%
  }%
  {%

    \usebeamerfont{author in sidebar}%
    \insertshortauthor[width=\swidth,center,respectlinebreaks]\par%
    \vskip1.25em%
  }%
  %\hbox to2cm{\hss\insertlogo\hss}
  \vskip1.25em%
  \insertverticalnavigation{\swidth}%
  \vfill
  \hbox to2cm{\hskip0.6cm\usebeamerfont{subsection in
      sidebar}\strut\usebeamercolor[fg]{subsection in
      sidebar}\insertframenumber-\inserttotalframenumber\hfill}%
  \vskip3pt%
}%
%%%%%%%%%%%%%%%%

\begin{document}
	\begin{frame}
	\maketitle
	\end{frame}

%%%%%%%%%%%%%%%
\begin{frame}
\frametitle{Contents}
\begin{itemize}
\item Setup
\begin{itemize}
\item \textbf{R}
\item RStudio
\item statnet package
\item datafiles for the class
\end{itemize}
\item Basic SNA Measures
\begin{itemize}
\item centrality measures
\item graph correlation
\item reciprocity
\item transitivity
\end{itemize}
\item Hypothesis testing
\begin{itemize}
\item for Node level indices
\begin{itemize}
\item General permutation tests
\item Quadratic Assignment Procedure
\item Network Autocorrelation Models
\end{itemize}
\item for Graph level indices
\begin{itemize}
\item Conditional Uniform Graph (CUG) Models
\end{itemize}
\end{itemize}
\end{itemize}
\end{frame}
%%%%%%%%%%%%%%%%
\section{Node Level Permutation}
\begin{frame}
\frametitle{Hypothesis Testing}
\end{frame}
%%%%%%%%%%%%%%%%
\begin{frame}
\frametitle{Relating Node level indices to covariates}
\begin{itemize}
\item Node Level Indices: centrality measures, brokerage, constraint
\pause
\item Node Covariates: measures of power, career advancement, gender -- really anything you want to study that varies at the node level
\end{itemize}
\end{frame}

%%%%%%%%%%%%%%%%
\begin{frame}
\frametitle{Emergent Multi-Organizational Networks (EMON) Dataset}
\begin{itemize}
\item 7 case studies of EMONs in the context of search and rescue activities from Drabek et. al. (1981)
\pause
\item Ties between organizations are self-reported levels of communication coded from 1 to 4 with 1 as most frequent
\end{itemize}
\end{frame}

%%%%%%%%%%%%%%%%
\begin{frame}
\frametitle{Emergent Multi-Organizational Networks (EMON) Dataset}
Attribute Data
\begin{itemize}
\item Command Rank Score (CRS): mean rank (reversed) for prominence in the command structure
\item Decision Rank Score (DRS): mean rank (reversed) for prominence in decision making process
\item Paid Staff: number of paid employees
\item Volunteer Staff: number of volunteer staff
\item Sponsorship: organization type (City, County, State, Federal, or Private) 
\end{itemize}
\end{frame}


%%%%%%%%%%%%%%%%

\begin{frame}
\frametitle{Correlation between DRS and Degree?}
\begin{columns}
\column{5cm}
\begin{itemize}
\uncover<2->{\item Subsample of Mutually Reported ``Continuous Communication" in Texas EMON}
\uncover<3->{\item Degree is shown in color (darker is bigger)}
\uncover<4->{\item DRS in size}
\uncover<5->{\item Empirical corelation $\rho=0.86$}
\end{itemize}
\column{5cm}
\includegraphics[width=5cm,trim=4.2cm 0cm 2.5cm 0cm, clip=true]{../images/texas1.jpeg}
\end{columns}
\end{frame}


%%%%%%%%%%%%%%%%

\begin{frame}
\frametitle{Correlation between DRS and Degree?}
\begin{columns}
\column{5cm}
\includegraphics[width=5cm,trim=4.2cm 2cm 2.5cm 3cm, clip=true]{../images/texas1.jpeg}
\begin{center}
$\rho=0.86$
\end{center}
\column{5cm}
\uncover<2->{
\includegraphics[width=5cm,trim=4.2cm 2cm 2.5cm 3cm, clip=true]{../images/texasSim2.jpeg}
\begin{center}
\uncover<3->{$\rho=-0.07$}
\end{center}
}
\end{columns}
\end{frame}

%%%%%%%%%%%%%%%%

\begin{frame}
\frametitle{Correlation between DRS and Degree?}
\begin{columns}
\column{5cm}
\includegraphics[width=5cm,trim=4.2cm 2cm 2.5cm 3cm, clip=true]{../images/texas1.jpeg}
\begin{center}
$\rho=0.86$
\end{center}
\column{5cm}
\includegraphics[width=2cm,trim=4.2cm 2cm 2.5cm 3cm, clip=true]{../images/texasSim1.jpeg}
$\rho=-0.07$
\uncover<2->{
\includegraphics[width=2cm,trim=4.2cm 2cm 2.5cm 3cm, clip=true]{../images/texasSim2.jpeg}
$\rho=-0.12$
}
\uncover<3->{
\includegraphics[width=2cm,trim=4.2cm 2cm 2.5cm 3cm, clip=true]{../images/texasSim3.jpeg}
$\rho=-0.39$
}
\end{columns}
\end{frame}

%%%%%%%%%%%%%%%%

\begin{frame}
\frametitle{Correlation between DRS and Degree?}
\begin{tikzpicture}
\node at (0,0) {\includegraphics[width=7cm,trim=0cm 0cm 0cm 0cm, clip=true]{../images/texasSimHist.jpg}};
\uncover<2->{
\draw [->,ultra thick] (3.5,1.8)--(2.8,1);
\node at (4,2){$\rho_{obs}=0.86$};
}
\uncover<3->{
\draw [->,ultra thick] (4.2,-1.3)--(2.8,-2.3);
\node at (5,-1){$Pr(\rho\geq\rho_{obs})$};
\node at (5, -1.5){$=3e-5$};
}
\uncover<4->{
\draw [->,ultra thick] (-2.5,2.5)--(-.8,.8);
\node at (-2.5,2.8){$Pr(\rho<\rho_{obs})=0.9999$};
}
\end{tikzpicture}
\end{frame}

%%%%%%%%%%%%%%%%
\begin{frame}
\frametitle{Regression?}
\begin{itemize}
\pause
\item Can use Node Level Indices as independent variables in a regression
\pause
\item Big assumption: \textit{\textbf{position}} predicts the \textit{\textbf{properties of those who hold them}}
\pause
\item Conditioning on NLI values, so dependence in accounted for \textit{\textbf{assuming no error in the network}}
\pause
\item NLIs as dependent variables more problematic due to autocorrelation
\end{itemize}
\end{frame}
%%%%%%%%%%%%%%%%%%%%
\begin{frame}
\frametitle{Code Time}
Sections 1-2.3
\end{frame}


%%%%%%%%%%%%%%%%
\section{Quadratic Assignment Procedure}
\begin{frame}
\frametitle{Quadratic Assignment Procedure}

\begin{columns}
\column{5.5cm} %first column
\includegraphics[width=5.5cm,trim=2cm 0.5cm 1.2cm 0.5cm, clip=true ]{../images/floMarriage.jpg}
\begin{center}
Marriage
\end{center}
\column{6cm} %second column
\includegraphics[width=5.5cm,,trim=2cm 0.5cm 1.2cm 0.5cm, clip=true]{../images/floBusiness.jpg}
\begin{center}
Business
\end{center}
\end{columns}
\end{frame}

%%%%%%%%%%%%%%%%
\begin{frame}
\frametitle{Quadratic Assignment Proceedure}

\begin{columns}
\column{5.5cm} %first column
\includegraphics[width=5.5cm,trim=2cm 0.5cm 1.2cm 0.5cm, clip=true ]{../images/floMarriage.jpg}
\begin{center}
Marriage
\end{center}
\column{6cm} %second column
\includegraphics[width=5.5cm,,trim=2cm 0.5cm 1.2cm 0.5cm, clip=true]{../images/floBusiness2.jpg}
\begin{center}
Business
\end{center}
\end{columns}
\end{frame}

%%%%%%%%%%%%%%%%
\begin{frame}
\frametitle{Graph Correlation}
\begin{itemize}
\pause
\item Simple way of comparing graphs on the same vertex set by element
\pause
\item $gcor \bigg(\begin{bmatrix}
1&1\\
1&0\\
\end{bmatrix},
\begin{bmatrix}
1&1\\
2&2\\
\end{bmatrix}
\bigg)=cor([1,1,1,0],[1,1,2,2])$
\end{itemize}
\end{frame}

%%%%%%%%%%%%%%%%
\begin{frame}
\frametitle{Do business ties coincide with marriages?}

\begin{columns}
\column{5.5cm} %first column
\includegraphics[width=5.5cm,trim=2cm 0.5cm 1.2cm 0.5cm, clip=true ]{../images/floMarriage.jpg}
\begin{center}
Marriage
\end{center}
\column{6cm} %second column
\includegraphics[width=5.5cm,,trim=2cm 0.5cm 1.2cm 0.5cm, clip=true]{../images/floBusiness2.jpg}
\begin{center}
Business
\end{center}
\end{columns}
\begin{center}
$\rho = 0.372$
\end{center}
\end{frame}

%%%%%%%%%%%%%%%%
\begin{frame}
\frametitle{Do business ties coincide with marriages?}

\begin{columns}
\column{5.5cm} %first column
\includegraphics[width=5.5cm,trim=2cm 0.5cm 1.2cm 0.5cm, clip=true ]{../images/floMarriage.jpg}
\begin{center}
Marriage
\end{center}
\column{6cm} %second column
\includegraphics[width=5.5cm,,trim=2cm 0.5cm 1.2cm 0.5cm, clip=true]{../images/floSim1.jpg}
\begin{center}
Business
\end{center}
\end{columns}
\pause
\begin{center}
$\rho = 0.169$
\end{center}
\end{frame}

%%%%%%%%%%%%%%%%
\begin{frame}
\frametitle{Do business ties coincide with marriages?}

\begin{columns}
\column{5.5cm} %first column
\includegraphics[width=5.5cm,trim=2cm 0.5cm 1.2cm 0.5cm, clip=true ]{../images/floMarriage.jpg}
\begin{center}
Marriage
\end{center}
\column{6cm} %second column
\includegraphics[width=5.5cm,,trim=2cm 0.5cm 1.2cm 0.5cm, clip=true]{../images/floSim2.jpg}
\begin{center}
Business
\end{center}
\end{columns}
\begin{center}
$\rho = -0.034$
\end{center}
\end{frame}

%%%%%%%%%%%%%%%%
\begin{frame}
\frametitle{Do business ties coincide with marriages?}

\begin{columns}
\column{5.5cm} %first column
\includegraphics[width=5.5cm,trim=2cm 0.5cm 1.2cm 0.5cm, clip=true ]{../images/floMarriage.jpg}
\begin{center}
Marriage
\end{center}
\column{6cm} %second column
\includegraphics[width=5.5cm,,trim=2cm 0.5cm 1.2cm 0.5cm, clip=true]{../images/floSim3.jpg}
\begin{center}
Business
\end{center}
\end{columns}
\begin{center}
$\rho = -0.101$
\end{center}
\end{frame}

%%%%%%%%%%%%%%%%

\begin{frame}
\frametitle{QAP Test}
\begin{tikzpicture}
\node at (0,0) {\includegraphics[width=7cm,trim=0cm 0cm 0cm 0cm, clip=true]{../images/floQap.jpg}};
\uncover<2->{
\draw [->,ultra thick] (3.5,1.8)--(2.8,1);
\node at (4,2){$\rho_{obs}=0.372$};
}
\uncover<3->{
\draw [->,ultra thick] (4.2,-.7)--(2.8,-1.8);
\node at (5,-.5){$Pr(\rho\geq\rho_{obs})$};
\node at (5, -1){$=.001$};
}
\uncover<4->{
\draw [->,ultra thick] (-2.5,2.5)--(-1.2,1.2);
\node at (-2.5,2.8){$Pr(\rho<\rho_{obs})=0.999$};
}
\end{tikzpicture}
\end{frame}
%%%%%%%%%%%%%%%%%%%%
\begin{frame}
\begin{center}
\LARGE Why can't we use the same permutation test?
\end{center}
\end{frame}
%%%%%%%%%%%%%%%%%%%%
\begin{frame}
\frametitle{QAP Test}
\includegraphics<1>[width=10cm]{../images/explainQAP1.png}
\includegraphics<2>[width=10cm]{../images/explainQAP2.png}
\includegraphics<3>[width=10cm]{../images/explainQAP3.png}

\end{frame}
%%%%%%%%%%%%%%%%%
\begin{frame}
\frametitle{Network Regression}
\begin{itemize}
\pause
\item Family of models predicting social ties
\pause
\begin{itemize}
\item Special case of standard OLS regression
\pause
\item Dependent variable is a network adjacency matrix
\end{itemize}
\pause
\item $\textbf{E}Y_{ij}=\beta_0+\beta_1X_{1ij}+\beta_2X_{2ij}+\dots+\beta_{\rho}X_{\rho ij}$
\pause
\begin{itemize}
\item Where $\textbf{E}$ is the expectation operator (analagous to ``mean" or ``average")
\pause
\item $Y_{ij}$ is the value from $i$ to $j$ on the dependent relation with adjacency matrix $Y$
\pause
\item $X_{kij}$ is the value of the \textit{k}th predictor for the $(i,j)$ ordered pair, and $\beta_0, \dots \beta_{\rho}$ are coefficients
\end{itemize}
\end{itemize}
\end{frame}
%%%%%%%%%%%%%%%%%%
\begin{frame}
\frametitle{Data Prep}
\begin{itemize}
\pause
\item Dependent variable is an adjacency matrix
\begin{itemize}
\pause
\item Standard case: dichotomous data
\pause
\item Valued case
\end{itemize}
\pause
\item Independent variables also in adjacency matrix form
\begin{itemize}
\pause
\item Always takes matrix form, but may be based on vector data
\pause
\item eg. simple adjacency matrix, sender/receiver effects, attribute differences, elements held in common
\end{itemize}
\end{itemize}
\end{frame}
%%%%%%%%%%%%%%%
\begin{frame}
\frametitle{Code Time}
Sections 2.4-2.5
\end{frame}

%%%%%%%%%%%%%%%%%%%%
\section{Network Autocorrelation}
\begin{frame}
\frametitle{Network Autocorrelation Models}
\begin{itemize}
\pause
\item Family of models for estimating how covariates relate to each other through ties
\begin{itemize}
\pause
\item Special case of standard OLS regression
\pause
\item Dependent variable is a vertex attribute
\pause
\end{itemize}
\item $y=(I-\Theta W)^{-1} (X\beta + (I-\psi Z)^{-1}v)$
\pause
\begin{itemize}
\item where $\Theta$ is the matrix for the Auto-Regressive weights
\pause
\item and $\psi$ is the matrix for the Moving Average weights
\end{itemize}
\end{itemize}
\end{frame}
%%%%%%%%%%%%%%%%%%%%
\begin{frame}
\frametitle{The Classical Regression Model}
\begin{tikzpicture}
\draw[white] (0,0) grid (10,6);
\draw [black] (3.5,2) rectangle (6.5,6);
\node (net) at (5,4.5) {\includegraphics[width=5cm]{../images/p1.png}};
\uncover<2->{\node [blue] at (5,2.5) {\LARGE{$X_i\beta$}};}
\uncover<4->{\draw [->,ultra thick, blue,dotted] (5,2.1)--(5,0.5);}
\uncover<3->{\node [blue] at (2,6) {\LARGE{$\epsilon_i$}};
\draw [->,ultra thick, blue,dotted] (2.25,6)--(4,5);}
\uncover<4->{\node [blue] at (5,0) {\LARGE{$y_i$}};}
\end{tikzpicture}
\end{frame}

%%%%%%%%%%%%%%%%%%%%
\begin{frame}
\frametitle{Adding Network AR Effects}
\begin{tikzpicture}
\draw[white] (0,0) grid (10,6);
\draw [black] (4,3) rectangle (6,6);
\node (net) at (5,5) {\includegraphics[width=3.5cm]{../images/p1.png}};
\node [blue] at (5,3.5) {\LARGE{$X_i\beta$}};
\draw [->,ultra thick, blue, dotted] (5,3.1)--(5,2.3);
\node [blue] at (3.4,6) {\LARGE{$\epsilon_i$}};
\draw [->,ultra thick, blue, dotted] (3.7,6)--(4.2,5.5);
\node [blue] at (5,2) {\LARGE{$y_i$}};

\draw [black] (1,2) rectangle (3,5);
\node (net) at (2,4) {\includegraphics[width=3.5cm]{../images/p4.png}};
\node [blue] at (2,2.5) {\LARGE{$X_j\beta$}};
\draw [->,ultra thick, blue, dotted] (2,2.1)--(2,1.4);
\node [blue] at (0.4,5) {\LARGE{$\epsilon_j$}};
\draw [->,ultra thick, blue, dotted] (.7,5)--(1.2,4.5);
\node [blue] at (2,1) {\LARGE{$y_j$}};
\uncover<2->{\draw [->, ultra thick, black] (3,3.5)--(4,4.5);}
\uncover<3->{\draw [->, ultra thick, red] (2.3, 1.2)--(4.45, 3.3);}

\draw [black] (7,2) rectangle (9,5);
\node (net) at (8,4) {\includegraphics[width=3.5cm]{../images/p3.png}};
\node [blue] at (8,2.5) {\LARGE{$X_k\beta$}};
\draw [->,ultra thick, blue, dotted] (8,2.1)--(8,1.4);
\node [blue] at (6.4,5) {\LARGE{$\epsilon_k$}};
\draw [->,ultra thick, blue, dotted] (6.7,5)--(7.2,4.5);
\node [blue] at (8,1) {\LARGE{$y_k$}};
\uncover<2->{\draw [->, ultra thick, black] (7,3.5)--(6,4.5);}
\uncover<3->{\draw [->, ultra thick, red] (7.7, 1.2)--(5.45, 3.3);}
\end{tikzpicture}
\end{frame}


%%%%%%%%%%%%%%%%%%%%
\begin{frame}
\frametitle{Adding Network MA Effects}
\begin{tikzpicture}
\draw[white] (0,0) grid (10,6);
\draw [black] (4,3) rectangle (6,6);
\node (net) at (5,5) {\includegraphics[width=3.5cm]{../images/p1.png}};
\node [blue] at (5,3.5) {\LARGE{$X_i\beta$}};
\draw [->,ultra thick, blue,dotted] (5,3.1)--(5,2.3);
\node [blue] at (3.4,6) {\LARGE{$\epsilon_i$}};
\draw [->,ultra thick, blue,dotted] (3.7,6)--(4.2,5.5);
\node [blue] at (5,2) {\LARGE{$y_i$}};

\draw [black] (1,2) rectangle (3,5);
\node (net) at (2,4) {\includegraphics[width=3.5cm]{../images/p4.png}};
\node [blue] at (2,2.5) {\LARGE{$X_j\beta$}};
\draw [->,ultra thick, blue,dotted] (2,2.1)--(2,1.4);
\node [blue] at (0.4,5) {\LARGE{$\epsilon_j$}};
\draw [->,ultra thick, blue,dotted] (.7,5)--(1.2,4.5);
\node [blue] at (2,1) {\LARGE{$y_j$}};
\draw [->, ultra thick, black] (3,3.5)--(4,4.5);
\uncover<2->{\draw [->, ultra thick, red] (0.5, 5.2)--(4.2, 4.7);}

\draw [black] (7,2) rectangle (9,5);
\node (net) at (8,4) {\includegraphics[width=3.5cm]{../images/p3.png}};
\node [blue] at (8,2.5) {\LARGE{$X_k\beta$}};
\draw [->,ultra thick, blue,dotted] (8,2.1)--(8,1.4);
\node [blue] at (6.4,5) {\LARGE{$\epsilon_k$}};
\draw [->,ultra thick, blue,dotted] (6.7,5)--(7.2,4.5);
\node [blue] at (8,1) {\LARGE{$y_k$}};
\draw [->, ultra thick, black] (7,3.5)--(6,4.5);
\uncover<2->{\draw [->, ultra thick, red] (6.1, 5)--(5.8, 4.7);}

\end{tikzpicture}
\end{frame}


%%%%%%%%%%%%%%%%%%%%
\begin{frame}
\frametitle{Network ARMA Model}
\begin{tikzpicture}
\draw[white] (0,0) grid (10,6);
\draw [black] (4,3) rectangle (6,6);
\node (net) at (5,5) {\includegraphics[width=3.5cm]{../images/p1.png}};
\node [blue] at (5,3.5) {\LARGE{$X_i\beta$}};
\draw [->,ultra thick, blue,dotted] (5,3.1)--(5,2.3);
\node [blue] at (3.4,6) {\LARGE{$\epsilon_i$}};
\draw [->,ultra thick, blue,dotted] (3.7,6)--(4.2,5.5);
\node [blue] at (5,2) {\LARGE{$y_i$}};

\draw [black] (1,2) rectangle (3,5);
\node (net) at (2,4) {\includegraphics[width=3.5cm]{../images/p4.png}};
\node [blue] at (2,2.5) {\LARGE{$X_j\beta$}};
\draw [->,ultra thick, blue,dotted] (2,2.1)--(2,1.4);
\node [blue] at (0.4,5) {\LARGE{$\epsilon_j$}};
\draw [->,ultra thick, blue,dotted] (.7,5)--(1.2,4.5);
\node [blue] at (2,1) {\LARGE{$y_j$}};
\draw [->, ultra thick, black] (3,3.5)--(4,4.5);
\draw [->, ultra thick, red] (0.5, 5.2)--(4.2, 4.7);
\draw [->, ultra thick, red] (2.3, 1.2)--(4.45, 3.3);

\draw [black] (7,2) rectangle (9,5);
\node (net) at (8,4) {\includegraphics[width=3.5cm]{../images/p3.png}};
\node [blue] at (8,2.5) {\LARGE{$X_k\beta$}};
\draw [->,ultra thick, blue,dotted] (8,2.1)--(8,1.4);
\node [blue] at (6.4,5) {\LARGE{$\epsilon_k$}};
\draw [->,ultra thick, blue,dotted] (6.7,5)--(7.2,4.5);
\node [blue] at (8,1) {\LARGE{$y_k$}};
\draw [->, ultra thick, black] (7,3.5)--(6,4.5);
\draw [->, ultra thick, red] (6.1, 5)--(5.8, 4.7);
\draw [->, ultra thick, red] (7.7, 1.2)--(5.45, 3.3);

\end{tikzpicture}
\end{frame}
%%%%%%%%%%%%%%%%%%
\begin{frame}
\frametitle{Network `Resonance'}
\begin{tikzpicture}
\draw[white] (0,0) grid (12,6);

\node (net) at (4,5) {\includegraphics[width=3.5cm]{../images/p1.png}};
\node (net) at (4,1) {\includegraphics[width=3.5cm]{../images/p4.png}};
\node (net) at (0,3) {\includegraphics[width=3.5cm]{../images/p3.png}};
\node (net) at (8,3) {\includegraphics[width=3.5cm]{../images/p2.png}};

\draw [<->, ultra thick, black] (1,3.2)--(3.3,5);
\draw [<->, ultra thick, black] (1,2.8)--(3.3,1);
\draw [<->, ultra thick, black] (4.7,5)--(7,3.2);
\draw [<->, ultra thick, black] (4.7,1)--(7,2.8);

\uncover<2->{
\node [blue] at (2.5,6) {\LARGE{$\epsilon_i$}};
\draw [->,ultra thick, blue,dotted] (2.8,6)--(3.3,5.5);
\node [blue] at (2.8,2) {\LARGE{$\epsilon_j$}};
\draw [->,ultra thick, blue,dotted] (3,2)--(3.5,1.7);
\node [blue] at (-1.2,4) {\LARGE{$\epsilon_k$}};
\draw [->,ultra thick, blue,dotted] (-.8,4)--(-.5,3.8);
\node [blue] at (6.8,4) {\LARGE{$\epsilon_l$}};
\draw [->,ultra thick, blue,dotted] (7.1,4)--(7.5,3.8);
}

\uncover<3->{
\draw [red, ultra thick] (-1.2,4) circle [radius=0.5];
}

\uncover<4->{
\draw [->, ultra thick, red, dashed] (1,3.5)--(3.3,5.3);
\draw [->, ultra thick, red, dashed] (1,2.5)--(3.3,.7);
}

\uncover<5->{
\draw [<-, thick, red, dashed] (1,3.8)--(3.3,5.6);
\draw [<-, thick, red, dashed] (1,2.2)--(3.3,.4);

\draw [->, thick, red, dashed] (4.8,5.3)--(6.9,3.6);
\draw [->, thick, red, dashed] (4.8,.7)--(6.9,2.4);
}

\uncover<6->{
\draw [->, red, dashed] (1,4.1)--(3.3,6);
\draw [->, red, dashed] (1,1.9)--(3.3,.1);

\draw [<-, red, dashed] (4.8,5.6)--(6.9,3.9);
\draw [<-, red, dashed] (4.8,.4)--(6.9,2.1);
}


\end{tikzpicture}
\end{frame}
%%%%%%%%%%%%%%%%%%%%%%%%%%
\begin{frame}
\frametitle{Inference with the Network Autocorrelation Model}
\begin{itemize}
\pause
\item Usually observe $\textbf{y}$, $\textbf{X}$, and $\textbf{Z}$ and/or $\textbf{Z}$, want to infer $\beta$, $\theta$, and $\phi$
\pause
\item Need each $\textbf{I}-\textbf{W}, \textbf{I}-\textbf{Z}$ invertible for solution to exist
\pause
\item error in disturbance autocorrelation, $v$, assumed as iid, $v_i~N(0,\sigma^2)$
\pause
\item Standard errors based on the inverse information matrix at the MLE
\pause
\item Compare models in the usual way (eg AIC, BIC)
\end{itemize}
\end{frame}
%%%%%%%%%%%%%%%%%%%%%%
\begin{frame}
\frametitle{Choosing the Weight Matrix}
\begin{itemize}
\pause
\item crucial modeling issue to choose the right form
\begin{itemize}
\pause
\item standard adjacency matrix
\pause
\item row-normalized adjancecy matrix
\pause
\item structural equivalence distance
\end{itemize}
\pause
\item Many suggestions given by Leenders 2002
\end{itemize}
\end{frame}
%%%%%%%%%%%%%%%%%%%%%%%
\begin{frame}
\frametitle{Data Prep}
\begin{itemize}
\pause
\item Dependent variable is a vertex attribute
\pause
\item Covariates are in matrix form with one column per attribute
\pause
\item Can include an intercept term by adding a column of 1s
\pause
\item Weight matrices for both AR and MA terms in matrix form
\pause
\item Can include multiple weight matrices (as a list) for both AR and MA
\end{itemize}
\end{frame}


%%%%%%%%%%%%%%%%%%%%%%%
\begin{frame}
\frametitle{Leenders 2002}
\includegraphics[width=7cm,keepaspectratio]{../images/louisiana.jpg}
\end{frame}
%%%%%%%%%%%%%%%%%%%%%%%
\begin{frame}
\frametitle{Leenders 2002}
\includegraphics[width=10cm,keepaspectratio]{../images/laNet.png}
\end{frame}
%%%%%%%%%%%%%%%%%%
\begin{frame}
\frametitle{Variables}
\begin{itemize}
\item Dependent variable: proportion of support in a parish for democratic presidential candidate Kennedy in the 1960 elections
\pause
\item Covariates: 
\begin{itemize}
\pause
\item $B$ is the percentage of African American residents in the parish
\pause
\item $C$ is the percentage of Catholic residents in the parish
\pause
\item $U$ is the percentage of the parish considered urban
\pause
\item $BPE$ is a measure of 'black political equality'
\end{itemize}
\pause
\item Weight matrix ($\rho$): simple contiguity network
\end{itemize}
\end{frame}
%%%%%%%%%%%%%%%%%%%%
\begin{frame}
\frametitle{Leenders 2002}
\includegraphics[width=10cm,keepaspectratio]{../images/leenders_tab3.jpg}
\end{frame}
%%%%%%%%%%%%%%%%%%%%%%%
\begin{frame}
\frametitle{Leenders 2002}
\includegraphics[width=10cm,keepaspectratio]{../images/leeders_tab5.png}
\end{frame}
%%%%%%%%%%%%%%%
\begin{frame}
\frametitle{Code Time}
Section 2.6
\end{frame}

%%%%%%%%%%%%%
\section{Baseline Models}
\begin{frame}
\frametitle{Baseline Models}
\begin{itemize}
\pause
\item treats social structure as maximally random given some fixed constraints
\pause
\item methodological premise from Mayhew
\begin{itemize}
\pause
\item identify potentially constraining factors
\pause
\item compare observed properties to baseline model
\pause
\item useful even when baseline model is not `realistic'
\end{itemize}
\end{itemize}
\end{frame}
%%%%%%%%%%%%%%%%%%%%
\begin{frame}
\frametitle{Types of Baseline Hypotheses}
\begin{tikzpicture}
\uncover<2->{
\draw [ultra thick, fill=red] (0,0) circle [radius=.2];
\draw [ultra thick, fill=yellow] (1,1) circle [radius=.2];
\draw [ultra thick, fill=green] (2,0) circle [radius=.2];
\draw [ultra thick, fill=blue] (1,-1) circle [radius=.2];
\draw [<->, ultra thick] (.15,.15)--(.85,.85);
\draw [->, ultra thick] (.2,0)--(1.8,0);
\draw [->, ultra thick] (.15,-.15)--(.85,-.85);
\node at (1,-2) {Empirical Network};
}
\uncover<3->{
\draw [thick, fill=red] (4,3) circle [radius=.1];
\draw [thick, fill=yellow] (4.5,3.5) circle [radius=.1];
\draw [thick, fill=blue] (5,3) circle [radius=.1];
\draw [thick, fill=green] (4.5,2.5) circle [radius=.1];
\draw [<->, thick] (4.05,3.05)--(4.45,3.45);
\draw [->, thick] (4.1,3)--(4.9,3);
\draw [->, thick] (4.05,2.95)--(4.45,2.55);
}

\uncover<4->{
\draw [thick, fill=red] (4,1) circle [radius=.1];
\draw [thick, fill=green] (4.5,1.5) circle [radius=.1];
\draw [thick, fill=blue] (5,1) circle [radius=.1];
\draw [thick, fill=yellow] (4.5,.5) circle [radius=.1];
\draw [<->, thick] (4.05,1.05)--(4.45,1.45);
\draw [->, thick] (4.1,1)--(4.9,1);
\draw [->, thick] (4.05,.95)--(4.45,.55);
}

\uncover<5->{
\draw [thick, fill=red] (4,-1) circle [radius=.1];
\draw [thick, fill=green] (4.5,-.5) circle [radius=.1];
\draw [thick, fill=yellow] (5,-1) circle [radius=.1];
\draw [thick, fill=blue] (4.5,-1.5) circle [radius=.1];
\draw [<->, thick] (4.05,-.95)--(4.45,-.55);
\draw [->, thick] (4.1,-1)--(4.9,-1);
\draw [->, thick] (4.05,-1.05)--(4.45,-1.45);

\draw [thick, fill=red] (4,-3) circle [radius=.1];
\draw [thick, fill=blue] (4.5,-2.5) circle [radius=.1];
\draw [thick, fill=green] (5,-3) circle [radius=.1];
\draw [thick, fill=yellow] (4.5,-3.5) circle [radius=.1];
\draw [<->, thick] (4.05,-2.95)--(4.45,-2.55);
\draw [->, thick] (4.1,-3)--(4.9,-3);
\draw [->, thick] (4.05,-3.05)--(4.45,-3.45);


\draw [thick, fill=red] (6,3) circle [radius=.1];
\draw [thick, fill=blue] (6.5,3.5) circle [radius=.1];
\draw [thick, fill=yellow] (7,3) circle [radius=.1];
\draw [thick, fill=green] (6.5,2.5) circle [radius=.1];
\draw [<->, thick] (6.05,3.05)--(6.45,3.45);
\draw [->, thick] (6.1,3)--(6.9,3);
\draw [->, thick] (6.05,2.95)--(6.45,2.55);

\draw [thick, fill=yellow] (6,1) circle [radius=.1];
\draw [thick, fill=red] (6.5,1.5) circle [radius=.1];
\draw [thick, fill=blue] (7,1) circle [radius=.1];
\draw [thick, fill=green] (6.5,.5) circle [radius=.1];
\draw [<->, thick] (6.05,1.05)--(6.45,1.45);
\draw [->, thick] (6.1,1)--(6.9,1);
\draw [->, thick] (6.05,.95)--(6.45,.55);

\draw [thick, fill=yellow] (6,-1) circle [radius=.1];
\draw [thick, fill=red] (6.5,-.5) circle [radius=.1];
\draw [thick, fill=green] (7,-1) circle [radius=.1];
\draw [thick, fill=blue] (6.5,-1.5) circle [radius=.1];
\draw [<->, thick] (6.05,-.95)--(6.45,-.55);
\draw [->, thick] (6.1,-1)--(6.9,-1);
\draw [->, thick] (6.05,-1.05)--(6.45,-1.45);

\draw [thick, fill=yellow] (6,-3) circle [radius=.1];
\draw [thick, fill=green] (6.5,-2.5) circle [radius=.1];
\draw [thick, fill=red] (7,-3) circle [radius=.1];
\draw [thick, fill=blue] (6.5,-3.5) circle [radius=.1];
\draw [<->, thick] (6.05,-2.95)--(6.45,-2.55);
\draw [->, thick] (6.1,-3)--(6.9,-3);
\draw [->, thick] (6.05,-3.05)--(6.45,-3.45);
\node at (8,-2) {$\dots$ etc};
}

\end{tikzpicture}
\end{frame}




%%%%%%%%%%%%%%%%%%%%
\begin{frame}
\frametitle{Types of Baseline Hypotheses}
\begin{tikzpicture}
\draw [ultra thick, fill=red] (0,0) circle [radius=.2];
\draw [ultra thick, fill=yellow] (1,1) circle [radius=.2];
\draw [ultra thick, fill=green] (2,0) circle [radius=.2];
\draw [ultra thick, fill=blue] (1,-1) circle [radius=.2];
\draw [<->, ultra thick] (.15,.15)--(.85,.85);
\draw [->, ultra thick] (.2,0)--(1.8,0);
\draw [->, ultra thick] (.15,-.15)--(.85,-.85);
\node at (1,-2) {Empirical Network};

\uncover<2->{
\draw [thick, fill=black] (4,3) circle [radius=.1];
\draw [thick, fill=black] (4.5,3.5) circle [radius=.1];
\draw [thick, fill=black] (5,3) circle [radius=.1];
\draw [thick, fill=black] (4.5,2.5) circle [radius=.1];
%\draw [<->, thick] (4.05,3.05)--(4.45,3.45);
%\draw [->, thick] (4.1,3)--(4.9,3);
%\draw [->, thick] (4.05,2.95)--(4.45,2.55);
}

\uncover<3->{
\draw [thick, fill=black] (4,1) circle [radius=.1];
\draw [thick, fill=black] (4.5,1.5) circle [radius=.1];
\draw [thick, fill=black] (5,1) circle [radius=.1];
\draw [thick, fill=black] (4.5,.5) circle [radius=.1];
%\draw [<->, thick] (4.05,1.05)--(4.45,1.45);
%\draw [->, thick] (4.1,1)--(4.9,1);
\draw [->, thick] (4.05,.95)--(4.45,.55);
}

\uncover<4->{
\draw [thick, fill=black] (4,-1) circle [radius=.1];
\draw [thick, fill=black] (4.5,-.5) circle [radius=.1];
\draw [thick, fill=black] (5,-1) circle [radius=.1];
\draw [thick, fill=black] (4.5,-1.5) circle [radius=.1];
%\draw [<->, thick] (4.05,-.95)--(4.45,-.55);
\draw [->, thick] (4.1,-1)--(4.9,-1);
\draw [->, thick] (4.05,-1.05)--(4.45,-1.45);

\draw [thick, fill=black] (4,-3) circle [radius=.1];
\draw [thick, fill=black] (4.5,-2.5) circle [radius=.1];
\draw [thick, fill=black] (5,-3) circle [radius=.1];
\draw [thick, fill=black] (4.5,-3.5) circle [radius=.1];
%\draw [<->, thick] (4.05,-2.95)--(4.45,-2.55);
\draw [->, thick] (4.55,-2.55)--(4.95,-2.95);
\draw [->, thick] (4.05,-3.05)--(4.45,-3.45);


\draw [thick, fill=black] (6,3) circle [radius=.1];
\draw [thick, fill=black] (6.5,3.5) circle [radius=.1];
\draw [thick, fill=black] (7,3) circle [radius=.1];
\draw [thick, fill=black] (6.5,2.5) circle [radius=.1];
\draw [<->, thick] (6.05,3.05)--(6.45,3.45);
\draw [->, thick] (6.1,3)--(6.9,3);
%\draw [->, thick] (6.05,2.95)--(6.45,2.55);

\draw [thick, fill=black] (6,1) circle [radius=.1];
\draw [thick, fill=black] (6.5,1.5) circle [radius=.1];
\draw [thick, fill=black] (7,1) circle [radius=.1];
\draw [thick, fill=black] (6.5,.5) circle [radius=.1];
\draw [->, thick] (6.05,1.05)--(6.45,1.45);
\draw [->, thick] (6.1,1)--(6.9,1);
\draw [->, thick] (6.55,1.45)--(6.95,1.05);

\draw [thick, fill=black] (6,-1) circle [radius=.1];
\draw [thick, fill=black] (6.5,-.5) circle [radius=.1];
\draw [thick, fill=black] (7,-1) circle [radius=.1];
\draw [thick, fill=black] (6.5,-1.5) circle [radius=.1];
\draw [<->, thick] (6.05,-.95)--(6.45,-.55);
\draw [->, thick] (6.1,-1)--(6.9,-1);
\draw [->, thick] (6.5,-.6)--(6.5,-1.4);

\draw [thick, fill=black] (6,-3) circle [radius=.1];
\draw [thick, fill=black] (6.5,-2.5) circle [radius=.1];
\draw [thick, fill=black] (7,-3) circle [radius=.1];
\draw [thick, fill=black] (6.5,-3.5) circle [radius=.1];
\draw [<->, thick] (6.1,-3)--(6.9,-3);
\draw [->, thick] (6.55,-2.55)--(6.95,-2.95);
\draw [->, thick] (6.05,-3.05)--(6.45,-3.45);
\node at (8,-2) {$\dots$ etc};
}

\end{tikzpicture}
\end{frame}
%%%%%%%%%%%%%%%%%%%%%%%%%%%%%

\begin{frame}
\frametitle{Types of Baseline Models}
\begin{itemize}
\pause
\item \textbf{Size:} given the number of individuals, all structures are equally likely
\pause
\item \textbf{Number of edges/probability of an edge:} given the number of individuals and interactions (aka Erd\"os-Renyi random graphs)
\pause
\item \textbf{Dyad census:} given number of individuals, mutuals, asymmetric, and null relationships
\pause
\item \textbf{Degree distribution:} given the number of individuals and each individual's outgoing/incoming ties
\pause
\item \textbf{Number of triangles:} not implemented due to complexity -- with ERGM, can condition on the \textit{expected} number of triangles
\end{itemize}
\end{frame}
%%%%%%%%%%%%%%%%
\begin{frame}
\frametitle{Method}
\begin{itemize}
\pause
\item Select a test statistic (graph correlation, reciprocity, transitivity\dots)
\pause
\item Select a baseline hypothesis (what you're conditioning on)
\pause
\item Simulate from the baseline hypothesis
\pause
\item For each simulation, recalculate the test statistic
\pause
\item Compare empirical value to null distribution, just as in standard statistical testing
\end{itemize}
\end{frame}
%%%%%%%%%%%%%%%%%%%%%%%%%%
\begin{frame}
\frametitle{Example}
Transitivity in the Hurricane Frederic EMON
\begin{columns}
\column{3cm}
\begin{itemize}
\uncover<3->{
\item $\rho = 0.475$
\item indicates that roughly half the time that $i\rightarrow j\rightarrow k$, $i\rightarrow k$
}
\end{itemize}
\column{8cm}
\uncover<2->{\includegraphics[width=8cm,trim=3cm 3cm 3cm 3cm, clip=true]{../images/hurricaneFrederic.png}}
\end{columns}
\end{frame}
%%%%%%%%%%%%%%%%%%%%%%%%%%%%
\begin{frame}
\frametitle{Example}
\includegraphics[width=8cm]{../images/cugSize.png}
\end{frame}
%%%%%%%%%%%%%%%%%%%%%%%%%%%%
\begin{frame}
\frametitle{Example}
\includegraphics[width=8cm]{../images/cugEdges.png}
\end{frame}
%%%%%%%%%%%%%%%%%%%%%%
\begin{frame}
\frametitle{Bodin and Tengo}
\uncover<1>{``Disentangling intangible social–ecological systems''}
\includegraphics<2>[width=9cm]{../images/bodinTengo1.png}
\includegraphics<3>[width=9cm]{../images/bodinTengo2.png}

\end{frame}
%%%%%%%%%%%%%%%%%%%%
\begin{frame}
\frametitle{Summary}
\begin{itemize}
\pause
\item Network indices as independent variables in regression
\pause
\item QAP regression (edges are the dependent variable)
\pause
\item Network Autocorrelation Model (vertex attribute is dependent variable)
\pause
\item CUG tests (network is dependent variable)
\end{itemize}
\end{frame}

%%%%%%%%%%%%%%%%%%%%%%
\begin{frame}
\frametitle{Code Time}
\begin{itemize}
\item the rest!  whew!
\end{itemize}
\end{frame}

\end{document}